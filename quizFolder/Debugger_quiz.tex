\documentclass{article}

\usepackage{amsmath}
\usepackage{amsfonts} % For math fonts.
\usepackage{amssymb} % For other math symbols not covered by amsmath.
\usepackage[pdftex]{graphicx} % For pictures, use \includegraphics[scale=decimal]{pic.png}; must be a .png file type.
\usepackage{multicol}
\usepackage{textcomp}
\usepackage[colorlinks = true, urlcolor = blue]{hyperref}
\usepackage{enumitem}
\usepackage{graphbox} 
\usepackage{subfig}
\usepackage{multicol}
\usepackage{nopageno}


\usepackage{tikz}
\usetikzlibrary{positioning, calc}
\usetikzlibrary{shapes.geometric,angles,quotes}
\usepackage{tikz-3dplot}


%page formatting
\usepackage{fullpage}
\setlength{\parindent}{0pt}


\newcommand{\tab}{\hspace*{0.25in}}
\newcommand{\csq}[1]{\reflectbox{''}#1''}  %This produces CS style quotes.
\newcommand{\csqt}[1]{\text{\reflectbox{''}#1''}}  %This produces CS style quotes as text.


\usepackage{listings}
\lstset
{ %Formatting for code in appendix
    language=Python,
    basicstyle=\footnotesize,
    numbers=left,
    stepnumber=1,
    showstringspaces=false,
    tabsize=2,
    breaklines=true,
    breakatwhitespace=false,
}


\begin{document}



%split_point

%\end{document}
Matt Priem \hfill Debugger quiz\\
section 1\\
\begin{enumerate}

%%%%%%%%%%%%%%%%%%%%%
	% Problem 6
%%%%%%%%%%%%%%%%%%%%%
	\item
		Write a function called \textit{flip\_flop} that takes a string as an argument 
		and returns a new word made up of the second half of the word first combined 
		with the first half of the word second.

		\textbf{Examples:}
		\begin{itemize}
			\item \textit{flip\_flop}(\csq{abcd}) $\rightarrow$ \csq{cdab} 
				(that is, \csq{cd} then \csq{ab} \dots even length)
			\item \textit{flip\_flop}(\csq{grapes}) $\rightarrow$ \csq{pesgra} 
				(that is, \csq{pes} then \csq{gra} \dots even length)
			\item \textit{flip\_flop}(\csq{abcde})$\rightarrow$ \csq{decab}
				(that is, \csq{de} then \csq{c} then \csq{ab} \dots odd length)
			\item \textit{flip\_flop}(\csq{cranberries})$\rightarrow$ \csq{rriesecranb}
				(that is, \csq{rries} then \csq{e} then \csq{cranb} \dots odd length)
		\end{itemize}

		\textbf{Debug this Solution:}\\
		\mbox{ \hspace*{0.25in}	\lstinputlisting[language=Python]{./code/debugger_p6_FlipFlop.py}}

%Error 1: if length // 2 == 0: should be if length % 2 == 0:
%Error 2: first_half = word[middle:] should be first_half = word[:middle]
%Error 3: num_1 == num_3 is repeated twice
\pagebreak




%%%%%%%%%%%%%%%%%%%%%
	% Problem 13
%%%%%%%%%%%%%%%%%%%%%
	\item 
		%https://edabit.com/challenge/6Pf5GGG6HnzbB95gf
		Write a \textbf{function} that returns a list with the factors of a given integer. 
		The argument of the function will be $num$ (integer to find factors for).

		\textbf{Examples:}		
		\begin{itemize}
			\item  find\_factors(12) $\rightarrow$ [1, 2, 3, 4, 6, 12], 
			\item  find\_factors(17) $\rightarrow$ [1, 17],
			\item  find\_factors(36) $\rightarrow$ [1, 2, 3, 4, 6, 9, 12, 18, 36]
		\end{itemize}

		\textbf{Debug this Solution:}\\
		\mbox{ \hspace*{0.25in}	\lstinputlisting[language=Python]{./code/debugger_p13_Factors.py}}

%Error 1: for i in range(1, num): should be for i in range(1, num+1):
%Error 2:  if num % i != 0: should be if num % i == 0:
%Error 3: factors.add(i) should be factors.append(i)
\pagebreak




\end{enumerate}
\pagebreak
Bart Simpson \hfill Debugger quiz\\
section 1\\
\begin{enumerate}

%%%%%%%%%%%%%%%%%%%%%
	% Problem 8
%%%%%%%%%%%%%%%%%%%%%
	\begin{minipage}{.6\textwidth}
		\item Create a \textit{Recipe} class.\\
		A \textit{Recipe} has
		\begin{itemize}
			\item name 
			\item cooking\_time
			%\item calores\_per\_serving
		\end{itemize}

		A \textit{Recipe} can do
		\begin{itemize}
			\item is\_quick\_meal
			%\item calculate\_calories\_total(servings)
		\end{itemize}
	\end{minipage}
		%
	\begin{minipage}{.4\textwidth}
		This class ``looks'' like 
				
		\vspace*{1em}
		\begin{tabular}{|l|}
			\hline Circle\\ \hline
			name\\ cooking\_time \\  \hline
			is\_quick\_meal\\ \hline
		\end{tabular}
	\end{minipage}

	\vspace*{2ex}
	Create a constructor method that initializes all instance variables.\\
	You should write getters and setters for each of the instance variables.\\
	Instantiate an instance of the class. You may pass any initial values of your choosing.	

	The is\_quick\_meal() method should return True if the cooking\_time is less than 30 minutes and False 
	if it takes 30 minutes or more.\\
		
		\textbf{Debug this Solution:}\\
		\mbox{ \hspace*{0.25in}	\lstinputlisting[language=Python]{./code/debugger_p8_RecipeClass.py}}

%Error 1: def __init__(name, cooking_time): should be def __init__(self, name, cooking_time):
%Error 2: return cooking_time should be return self.cooking_time
%Error 3: return self.cooking_time == 30 should be return self.cooking_time < 30
\pagebreak




%%%%%%%%%%%%%%%%%%%%%
	% Problem 11
%%%%%%%%%%%%%%%%%%%%%
	\item 	
		%https://edabit.com/challenge/yL5WmWTCNwwb4GnR7
		In each input list, every number repeats at least once, except for two. Write a \textbf{function} 
		that takes an array $numbers$ and returns the two unique numbers.

		\textbf{Examples:}		
		\begin{itemize}
			\item  return\_unique([1, 9, 8, 8, 7, 6, 1, 6]) $\rightarrow$ [9, 7],
			\item  return\_unique([5, 5, 2, 4, 4, 4, 9, 9, 9, 1]) $\rightarrow$ [2, 1],
			\item  return\_unique([9, 5, 6, 8, 7, 7, 1, 1, 1, 1, 1, 9, 8]) $\rightarrow$ [5, 6]
		\end{itemize}

		\textbf{Debug this Solution:}\\
		\mbox{ \hspace*{0.25in}	\lstinputlisting[language=Python]{./code/debugger_p11_TwoUniqueNumbers.py}}

%Error 1: for num in range(len(numbers)) should be for num in numbers
%Error 2: code in if and else statements should be swapped.
%Error 3: number_dicitonary.values() should be number_dicitonary
\pagebreak



\end{enumerate}
\pagebreak
Lone Star \hfill Debugger quiz\\
section 2\\
\begin{enumerate}

%%%%%%%%%%%%%%%%%%%%%
	% Problem 12
%%%%%%%%%%%%%%%%%%%%%
	\begin{minipage}{.6\textwidth}
	\item Create a \textit{Vector} class.\\
		A \textit{Vector} has
		\begin{itemize}
			\item x\_direction 
			\item y\_direction
		\end{itemize}
		
		A \textit{Vector} can do
		\begin{itemize}
			\item get\_magnitude
		\end{itemize}
	\end{minipage}
		%
	\begin{minipage}{.4\textwidth}
		This class ``looks'' like 
				
		\vspace*{1em}
		\begin{tabular}{|l|}
			\hline Vector\\ \hline
			x\_direction\\ y\_direction\\ \hline
			get\_magnitude\\  \hline
		\end{tabular}
	\end{minipage}

	\vspace*{2ex}
	Create a constructor method that initializes all instance variables.\\
	You should write getters and setters for each of the instance variables.\\
	Instantiate an instance of the class. You may pass any initial values of your choosing.
	
	Hint: magnitude is calculated as $\sqrt{x^2 + y^2}$.

		\textbf{Debug this Solution:}\\
		\mbox{ \hspace*{0.25in}	\lstinputlisting[language=Python]{./code/debugger_p12_Vector.py}}

%Error 1: def __init__(x_direction, y_direction): should be def __init__(self, x_direction, y_direction):
%Error 2: return self.y_direction should be return self.x_direction
%Error 3: sqrt should be math.sqrt and import math OR from math import sqrt
\pagebreak




%%%%%%%%%%%%%%%%%%%%%
	% Problem 10
%%%%%%%%%%%%%%%%%%%%%
	\item
		YouTube currently displays a like and a dislike button, allowing you to express your opinions 
		about particular content. 
		It's set up in such a way that you cannot like and dislike a video at the same time.
		There are two other interesting rules to be noted about the interface:
		\begin{enumerate}
			\item Pressing a button, which is already active, will undo your press.
			\item If you press the like button after pressing the dislike button, the like button overwrites 
				the previous \csq{dislike} state. The same is true for the other way round.
		\end{enumerate}
		Write a \textbf{function} that takes in a list of button inputs $events$ and returns the final state.

		\textbf{Examples:}		
		\begin{itemize}
			\item  like\_or\_dislike([\csq{dislike}]) $\rightarrow$ \csq{dislike},
			\item  like\_or\_dislike([\csq{like}, \csq{like}]) $\rightarrow$ \csq{nothing},
			\item  like\_or\_dislike([\csq{dislike}, \csq{like}]) $\rightarrow$ \csq{like},
			\item  like\_or\_dislike([\csq{like}, \csq{dislike}, \csq{dislike}]) $\rightarrow$ \csq{nothing},
		\end{itemize}

		\textbf{Debug this Solution:}\\
		\mbox{ \hspace*{0.25in}	\lstinputlisting[language=Python]{./code/debugger_p10_YouTube.py}}

%Error 1: initial state = "like" should be state = "nothing"
%Error 2: for event in range(events) should be for event in events
%Error 3: if event != state: should be if event == state:
\pagebreak



\end{enumerate}
\pagebreak
Dot Matrix \hfill Debugger quiz\\
section 3\\
\begin{enumerate}

%%%%%%%%%%%%%%%%%%%%%
	% Problem 8
%%%%%%%%%%%%%%%%%%%%%
	\begin{minipage}{.6\textwidth}
		\item Create a \textit{Recipe} class.\\
		A \textit{Recipe} has
		\begin{itemize}
			\item name 
			\item cooking\_time
			%\item calores\_per\_serving
		\end{itemize}

		A \textit{Recipe} can do
		\begin{itemize}
			\item is\_quick\_meal
			%\item calculate\_calories\_total(servings)
		\end{itemize}
	\end{minipage}
		%
	\begin{minipage}{.4\textwidth}
		This class ``looks'' like 
				
		\vspace*{1em}
		\begin{tabular}{|l|}
			\hline Circle\\ \hline
			name\\ cooking\_time \\  \hline
			is\_quick\_meal\\ \hline
		\end{tabular}
	\end{minipage}

	\vspace*{2ex}
	Create a constructor method that initializes all instance variables.\\
	You should write getters and setters for each of the instance variables.\\
	Instantiate an instance of the class. You may pass any initial values of your choosing.	

	The is\_quick\_meal() method should return True if the cooking\_time is less than 30 minutes and False 
	if it takes 30 minutes or more.\\
		
		\textbf{Debug this Solution:}\\
		\mbox{ \hspace*{0.25in}	\lstinputlisting[language=Python]{./code/debugger_p8_RecipeClass.py}}

%Error 1: def __init__(name, cooking_time): should be def __init__(self, name, cooking_time):
%Error 2: return cooking_time should be return self.cooking_time
%Error 3: return self.cooking_time == 30 should be return self.cooking_time < 30
\pagebreak




%%%%%%%%%%%%%%%%%%%%%
	% Problem 7
%%%%%%%%%%%%%%%%%%%%%
	\item 
		The hamming distance is the number of characters that differ between two strings. 
		Write a function named hamming\_distance that takes two strings as arguments and 
		returns the hamming distance.
		
		\textbf{Examples:}
		\begin{itemize}
			\item \textit{hamming\_distance}(\csq{river}, \csq{rover}) $\rightarrow$ 1 
			\item \textit{hamming\_distance}(\csq{cat}, \csq{dog}) $\rightarrow$ 3 
			\item \textit{hamming\_distance}(\csq{cat}, \csq{hat}) $\rightarrow$ 1
			\item \textit{hamming\_distance}(\csq{cat}, \csq{banana}) $\rightarrow$ 
				\csq{Strings must be of equal length.}
		\end{itemize}		
		
		\textbf{Debug this Solution:}\\
		\mbox{ \hspace*{0.25in}	\lstinputlisting[language=Python]{./code/debugger_p7_HammingDistance.py}}

%Error 1: distance = 1 should be distance = 0
%Error 2: for i in range(len(str1) -1): should be for i in range(len(str1) ):
%Error 3: if str1[i] == str2[i]: should be if str1[i] != str2[i]:
\pagebreak



\end{enumerate}
\pagebreak
Alfred Yankovic \hfill Debugger quiz\\
section 2\\
\begin{enumerate}

%%%%%%%%%%%%%%%%%%%%%
	% Problem 13
%%%%%%%%%%%%%%%%%%%%%
	\item 
		%https://edabit.com/challenge/6Pf5GGG6HnzbB95gf
		Write a \textbf{function} that returns a list with the factors of a given integer. 
		The argument of the function will be $num$ (integer to find factors for).

		\textbf{Examples:}		
		\begin{itemize}
			\item  find\_factors(12) $\rightarrow$ [1, 2, 3, 4, 6, 12], 
			\item  find\_factors(17) $\rightarrow$ [1, 17],
			\item  find\_factors(36) $\rightarrow$ [1, 2, 3, 4, 6, 9, 12, 18, 36]
		\end{itemize}

		\textbf{Debug this Solution:}\\
		\mbox{ \hspace*{0.25in}	\lstinputlisting[language=Python]{./code/debugger_p13_Factors.py}}

%Error 1: for i in range(1, num): should be for i in range(1, num+1):
%Error 2:  if num % i != 0: should be if num % i == 0:
%Error 3: factors.add(i) should be factors.append(i)
\pagebreak





%%%%%%%%%%%%%%%%%%%%%
	% Problem 15
%%%%%%%%%%%%%%%%%%%%%
	\item
		(Game: Odd or Even)  Write a \textbf{function} that lets the user guess whether a randomly 
		generated number is odd or even.  The function randomly generates an integer between 0 and 9 
		(inclusive) and returns whether the user's guess is correct or incorrect. The argument for 
		the function will be $guess$ (the user's guess, either \csq{odd} or \csq{even}), if no 
		argument is provided then the \textbf{default} guess should be even.\\
		Hint: Use the following lines of code to create the function.
		\begin{verbatim}
		    from random import randint
		    value = randint(0,9) #picks a random integer between 0-9 inclusive
		\end{verbatim}
		\textbf{Examples:}
		\begin{itemize}
			\item  guess( ) $\rightarrow$ \csq{Correct!} (if random value is even) 
				or \csq{Incorrect!} (if random value is odd) 
			\item  guess(\csq{odd})$\rightarrow$\csq{Correct!} (if random value is odd) 
				or \csq{Incorrect!} (if random value is even)
			\item  guess(\csq{even}) $\rightarrow$ \csq{Correct!} (if random value is even) 
				or \csq{Incorrect!} (if random value is odd) 
		\end{itemize}

		\textbf{Debug this Solution:}\\
		\mbox{ \hspace*{0.25in}	\lstinputlisting[language=Python]{./code/debugger_p15_EvenOrOdd.py}}


%Error 1: from random import randominteger should be from random import randomint
%Error 2: def guess(guess="odd"): should be def guess(guess="even"):
%Error 3: if value // 2 == 0: should be if value % 2 == 0:
\pagebreak



\end{enumerate}
\pagebreak
\end{document}