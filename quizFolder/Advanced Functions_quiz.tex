\documentclass{article}

\usepackage{amsmath}
\usepackage{amsfonts} % For math fonts.
\usepackage{amssymb} % For other math symbols not covered by amsmath.
\usepackage[pdftex]{graphicx} % For pictures, use \includegraphics[scale=decimal]{pic.png}; must be a .png file type.
\usepackage{multicol}
\usepackage{textcomp}
\usepackage[colorlinks = true, urlcolor = blue]{hyperref}
\usepackage{enumitem}
\usepackage{graphbox} 
\usepackage{subfig}
\usepackage{multicol}
\usepackage{nopageno}


\usepackage{tikz}
\usetikzlibrary{positioning, calc}
\usetikzlibrary{shapes.geometric,angles,quotes}
\usepackage{tikz-3dplot}


%page formatting
\usepackage{fullpage}
\setlength{\parindent}{0pt}


\newcommand{\tab}{\hspace*{0.25in}}
\newcommand{\csq}[1]{\reflectbox{''}#1''}  %This produces CS style quotes.
\newcommand{\csqt}[1]{\text{\reflectbox{''}#1''}}  %This produces CS style quotes as text.


\usepackage{listings}
\lstset
{ %Formatting for code in appendix
    language=Python,
    basicstyle=\footnotesize,
    numbers=left,
    stepnumber=1,
    showstringspaces=false,
    tabsize=2,
    breaklines=true,
    breakatwhitespace=false,
}


\begin{document}



%split_point

%\end{document}
Matt Priem \hfill Advanced Functions quiz\\
section 1\\
\begin{enumerate}
	\item 
		Luke Skywalker has friends and family, but he is getting older and having trouble 
		remembering them all.  Write a \textbf{function} that will return the relation 
		defined in the table below. The arguments to the function will be $name$ 
		(name of the person related to Luke), if no argument is provided then the 
		\textbf{default} should be nothing. That is, the empty word \csq{ }. \\ 
		\begin{center}
		\begin{tabular}{|l|l|} \hline
			Person 		& Relation \\ \hline \hline
			Darth Vader	& Father \\ \hline
			Leia		& Sister \\ \hline
			Han			& Brother in law\\ \hline
			R2D2		& Droid \\ \hline
		\end{tabular}\\ \hspace*{1in} *If he types any other name, return \csq{unknown}.
		\end{center}
		\textbf{Examples:}		
		\begin{itemize}
			\item  find\_relation(\csq{Darth Vader}) $\rightarrow$ \csq{Father}, 
			\item  find\_relation(\csq{R2D2}) $\rightarrow$ \csq{Droid}, 
			\item  find\_relation(\csq{Jabba the Hutt}) $\rightarrow$ \csq{Unknown}
			\item  find\_relation( ) $\rightarrow$ \csq{Unknown}
		\end{itemize}


	\item
		Write a \textbf{function} that takes 3 numbers as arguments, $num\_1$ (first number), 
		$num\_2$ (second number), and $num\_3$ (third number).  $num\_1$ should be mandatory.
		If no arguments are provided for $num\_2$ or $num\_3$ then use 5 for $num\_2$ and 
		25 for $num\_3$.
		Return a list of the integers in ascending order. \\
		You may \textbf{not} use the built-in functions \textit{max}(), \textit{min}(), 
		\textit{sort}(), or \textit{sorted}().
		
	\textbf{Examples:}
	\begin{itemize}
		\item  ascending\_order(2, 3, 1) $\rightarrow$ [1, 2, 3], 
		\item  ascending\_order(10, 1) $\rightarrow$ [1, 10, 25], 
		\item  ascending\_order(50) $\rightarrow$ [5, 25, 50] 
	\end{itemize}


	\item 
		Given a positive integer $n$, the following rules will always create a sequence that 
		ends with 1, called Hailstone Sequence:
		\begin{enumerate}
			\item If $n$ is even, divide by 2
			\item If $n$ is odd, multiply by 3 and add 1 (i.e. $3n+1$)
			\item Continue until $n$ is 1
		\end{enumerate}
		Write a \textbf{function} that prints the hailstone sequence starting at $n$. 
		The argument to the function will be $n$ (the integer to start the sequence from), 
		if no argument is provided then the \textbf{default} should be 40.
		\textbf{Examples:}		
		\begin{itemize}
			\item  hailstone\_seq(25) $\rightarrow$ 25, 76, 38, 19, 58 ... 8, 4, 2, 1, 
			\item  hailstone\_seq(40) $\rightarrow$ 40, 20, 10, 5, 16, 8, 4, 2, 1
			\item  hailstone\_seq( ) $\rightarrow$ 40, 20, 10, 5, 16, 8, 4, 2, 1
		\end{itemize}


\end{enumerate}
\pagebreak
Bart Simpson \hfill Advanced Functions quiz\\
section 1\\
\begin{enumerate}
	\item 
		Write a \textbf{function} to create a game of Rock, Paper, Scissors. The function will 
		return the winner of the game played by two players.
		The arguments to the function will be $player1$ (the first player's choice) and $player2$ 
		(the second player's choice), if no argument is provided then the \textbf{default} for 
		either player should be Rock.\\
		Print the winner according to the following rules. 
		\begin{itemize}
			\item Rock beats Scissors
			\item Scissors beats Paper
			\item Paper beats Rock
		\end{itemize}		
		\textbf{Examples:}		
		\begin{itemize}
			\item  find\_winner(\csq{Rock}, \csq{Paper}) $\rightarrow$ \csq{Player 2 wins!}, 
			\item  find\_winner(\csq{Scissors}, \csq{Paper}) $\rightarrow$ \csq{Player 1 wins!}, 
			\item  find\_winner(\csq{Rock}, \csq{Rock}) $\rightarrow$ \csq{It's a tie!}
			\item  find\_winner(\csq{Rock}) $\rightarrow$ \csq{It's a tie!}
			\item  find\_winner( ) $\rightarrow$ \csq{It's a tie!}
			\item  find\_winner(\csq{Scissors}) $\rightarrow$ \csq{Player 2 wins!}
		\end{itemize}


	\item
		Write a \textbf{function} that takes 3 numbers as arguments, $num\_1$ (first number), 
		$num\_2$ (second number), and $num\_3$ (third number).  $num\_1$ should be mandatory.
		If no arguments are provided for $num\_2$ or $num\_3$ then use 5 for $num\_2$ and 
		25 for $num\_3$.
		Return a list of the integers in ascending order. \\
		You may \textbf{not} use the built-in functions \textit{max}(), \textit{min}(), 
		\textit{sort}(), or \textit{sorted}().
		
	\textbf{Examples:}
	\begin{itemize}
		\item  ascending\_order(2, 3, 1) $\rightarrow$ [1, 2, 3], 
		\item  ascending\_order(10, 1) $\rightarrow$ [1, 10, 25], 
		\item  ascending\_order(50) $\rightarrow$ [5, 25, 50] 
	\end{itemize}


	\item 
		Luke Skywalker has friends and family, but he is getting older and having trouble 
		remembering them all.  Write a \textbf{function} that will return the relation 
		defined in the table below. The arguments to the function will be $name$ 
		(name of the person related to Luke), if no argument is provided then the 
		\textbf{default} should be nothing. That is, the empty word \csq{ }. \\ 
		\begin{center}
		\begin{tabular}{|l|l|} \hline
			Person 		& Relation \\ \hline \hline
			Darth Vader	& Father \\ \hline
			Leia		& Sister \\ \hline
			Han			& Brother in law\\ \hline
			R2D2		& Droid \\ \hline
		\end{tabular}\\ \hspace*{1in} *If he types any other name, return \csq{unknown}.
		\end{center}
		\textbf{Examples:}		
		\begin{itemize}
			\item  find\_relation(\csq{Darth Vader}) $\rightarrow$ \csq{Father}, 
			\item  find\_relation(\csq{R2D2}) $\rightarrow$ \csq{Droid}, 
			\item  find\_relation(\csq{Jabba the Hutt}) $\rightarrow$ \csq{Unknown}
			\item  find\_relation( ) $\rightarrow$ \csq{Unknown}
		\end{itemize}


\end{enumerate}
\pagebreak
Lone Star \hfill Advanced Functions quiz\\
section 2\\
\begin{enumerate}
	\item 
		In an Ancient Kingdom, the currency consists of bronze coins, silver coins, and gold coins.  
		There are 20 bronze coins in one silver coin and 15 silver coins in one gold coin.  Write a 	
		\textbf{function} that will return a converted amount of bronze coins into the fewest amount 
		of coins possible.  Only return a string with the non-zero values, meaning don't return 
		something similar to ``0 silver coins''. The argument for the function will be 
		$bronze\_coins$ (how many bronze coins to convert)., if no argument is provided then the 
		\textbf{default} should be 900 bronze coins 

		\textbf{Examples:}
		\begin{itemize}
			\item  convert\_bronze(32) $\rightarrow$ \csq{1 silver 12 bronze}, 
			\item  convert\_bronze(544) $\rightarrow$ \csq{1 gold 4 silver 4 bronze}, 
			\item  convert\_bronze(903) $\rightarrow$ \csq{3 gold 3 bronze}\\
				Note: Do \textbf{not} output 3 gold 0 silver 3 bronze.
			\item  convert\_bronze() $\rightarrow$ \csq{3 gold}\\
				Note: Do \textbf{not} output 3 gold 0 silver 0 bronze.
		\end{itemize}


	\item
		Write a \textbf{function} that takes 3 numbers as arguments, $num\_1$ (first number), 
		$num\_2$ (second number), and $num\_3$ (third number).  $num\_1$ should be mandatory.
		If no arguments are provided for $num\_2$ or $num\_3$ then use 5 for $num\_2$ and 
		25 for $num\_3$.
		Return a list of the integers in ascending order. \\
		You may \textbf{not} use the built-in functions \textit{max}(), \textit{min}(), 
		\textit{sort}(), or \textit{sorted}().
		
	\textbf{Examples:}
	\begin{itemize}
		\item  ascending\_order(2, 3, 1) $\rightarrow$ [1, 2, 3], 
		\item  ascending\_order(10, 1) $\rightarrow$ [1, 10, 25], 
		\item  ascending\_order(50) $\rightarrow$ [5, 25, 50] 
	\end{itemize}


	\item 
		Write a \textbf{function} to create a game of Rock, Paper, Scissors. The function will 
		return the winner of the game played by two players.
		The arguments to the function will be $player1$ (the first player's choice) and $player2$ 
		(the second player's choice), if no argument is provided then the \textbf{default} for 
		either player should be Rock.\\
		Print the winner according to the following rules. 
		\begin{itemize}
			\item Rock beats Scissors
			\item Scissors beats Paper
			\item Paper beats Rock
		\end{itemize}		
		\textbf{Examples:}		
		\begin{itemize}
			\item  find\_winner(\csq{Rock}, \csq{Paper}) $\rightarrow$ \csq{Player 2 wins!}, 
			\item  find\_winner(\csq{Scissors}, \csq{Paper}) $\rightarrow$ \csq{Player 1 wins!}, 
			\item  find\_winner(\csq{Rock}, \csq{Rock}) $\rightarrow$ \csq{It's a tie!}
			\item  find\_winner(\csq{Rock}) $\rightarrow$ \csq{It's a tie!}
			\item  find\_winner( ) $\rightarrow$ \csq{It's a tie!}
			\item  find\_winner(\csq{Scissors}) $\rightarrow$ \csq{Player 2 wins!}
		\end{itemize}


\end{enumerate}
\pagebreak
Dot Matrix \hfill Advanced Functions quiz\\
section 3\\
\begin{enumerate}
	\item 
		Write a \textbf{function} that returns the number of copies of the same number. 
		The arguments for the function will be $num\_1$ (first number), $num\_2$ (second number), 
		and $num\_3$ (third number), if no argument is provided then the \textbf{default} for all 
		3 values should be 0.\\

		\textbf{Examples:}		
		\begin{itemize}
			\item  count\_duplicates(2, 3, 2) $\rightarrow$ \csq{There are 2 of the same number}, 
			\item  count\_duplicates(4, 4, 4) $\rightarrow$ \csq{There are 3 of the same number}, 
			\item  count\_duplicates(1, 2, 3) $\rightarrow$ \csq{Each number is unique} 
			\item  count\_duplicates(1) $\rightarrow$ \csq{There are 2 of the same number} 
			\item  count\_duplicates(0) $\rightarrow$ \csq{There are 3 of the same number} 
		\end{itemize}


	\item 
		Luke Skywalker has friends and family, but he is getting older and having trouble 
		remembering them all.  Write a \textbf{function} that will return the relation 
		defined in the table below. The arguments to the function will be $name$ 
		(name of the person related to Luke), if no argument is provided then the 
		\textbf{default} should be nothing. That is, the empty word \csq{ }. \\ 
		\begin{center}
		\begin{tabular}{|l|l|} \hline
			Person 		& Relation \\ \hline \hline
			Darth Vader	& Father \\ \hline
			Leia		& Sister \\ \hline
			Han			& Brother in law\\ \hline
			R2D2		& Droid \\ \hline
		\end{tabular}\\ \hspace*{1in} *If he types any other name, return \csq{unknown}.
		\end{center}
		\textbf{Examples:}		
		\begin{itemize}
			\item  find\_relation(\csq{Darth Vader}) $\rightarrow$ \csq{Father}, 
			\item  find\_relation(\csq{R2D2}) $\rightarrow$ \csq{Droid}, 
			\item  find\_relation(\csq{Jabba the Hutt}) $\rightarrow$ \csq{Unknown}
			\item  find\_relation( ) $\rightarrow$ \csq{Unknown}
		\end{itemize}


	\item 
		(Game: heads or tails)  Write a \textbf{function} that lets the user guess whether the flip of a coin 
		results in heads or tails. The function randomly generates an integer 0 or 1, which 
		represents head or tail. The function returns if the guess is correct or incorrect. The argument for the function will be $guess$ 
		(the guess of the user, 0 for heads and 1 for tails), if no argument is provided then the \textbf{default} should be 0 for heads.\\
		Hint: Use the following lines of code to create the function.
		\begin{verbatim}
		    from random import randint
		    value = randint(0,1) #picks a random integer. Either 0 or 1.
		\end{verbatim}
		\textbf{Examples:}
		\begin{itemize}
			\item  toss\_coin( ) $\rightarrow$ \csq{Correct!} (if the random value is 0) or 
				\csq{Incorrect!} (if the random value is 1), 
			\item  toss\_coin(0) $\rightarrow$ \csq{Correct!} (if the random value is 0) or 
				\csq{Incorrect!} (if the random value is 1), 
			\item  toss\_coin(1) $\rightarrow$ \csq{Correct!} (if the random value is 1) or 
				\csq{Incorrect!} (if the random value is 0) 
		\end{itemize}

\end{enumerate}
\pagebreak
Alfred Yankovic \hfill Advanced Functions quiz\\
section 2\\
\begin{enumerate}
	\item 
		Luke Skywalker has friends and family, but he is getting older and having trouble 
		remembering them all.  Write a \textbf{function} that will return the relation 
		defined in the table below. The arguments to the function will be $name$ 
		(name of the person related to Luke), if no argument is provided then the 
		\textbf{default} should be nothing. That is, the empty word \csq{ }. \\ 
		\begin{center}
		\begin{tabular}{|l|l|} \hline
			Person 		& Relation \\ \hline \hline
			Darth Vader	& Father \\ \hline
			Leia		& Sister \\ \hline
			Han			& Brother in law\\ \hline
			R2D2		& Droid \\ \hline
		\end{tabular}\\ \hspace*{1in} *If he types any other name, return \csq{unknown}.
		\end{center}
		\textbf{Examples:}		
		\begin{itemize}
			\item  find\_relation(\csq{Darth Vader}) $\rightarrow$ \csq{Father}, 
			\item  find\_relation(\csq{R2D2}) $\rightarrow$ \csq{Droid}, 
			\item  find\_relation(\csq{Jabba the Hutt}) $\rightarrow$ \csq{Unknown}
			\item  find\_relation( ) $\rightarrow$ \csq{Unknown}
		\end{itemize}


	\item 
		Write a \textbf{function} that returns the factors of a given integer. 
		The argument of the function will be $num$ (integer to find factors for), 
		if no argument is provided then the \textbf{default} should be 36.

		\textbf{Examples:}		
		\begin{itemize}
			\item  find\_factors(12) $\rightarrow$ 1, 2, 3, 4, 6, 12, 
			\item  find\_factors(17) $\rightarrow$ 1, 17,
			\item  find\_factors(36) $\rightarrow$ 1, 2, 3, 4, 6, 9, 12, 18, 36
			\item  find\_factors( ) $\rightarrow$ 1, 2, 3, 4, 6, 9, 12, 18, 36
		\end{itemize}


	\item 
		Given a positive integer $n$, the following rules will always create a sequence that 
		ends with 1, called Hailstone Sequence:
		\begin{enumerate}
			\item If $n$ is even, divide by 2
			\item If $n$ is odd, multiply by 3 and add 1 (i.e. $3n+1$)
			\item Continue until $n$ is 1
		\end{enumerate}
		Write a \textbf{function} that prints the hailstone sequence starting at $n$. 
		The argument to the function will be $n$ (the integer to start the sequence from), 
		if no argument is provided then the \textbf{default} should be 40.
		\textbf{Examples:}		
		\begin{itemize}
			\item  hailstone\_seq(25) $\rightarrow$ 25, 76, 38, 19, 58 ... 8, 4, 2, 1, 
			\item  hailstone\_seq(40) $\rightarrow$ 40, 20, 10, 5, 16, 8, 4, 2, 1
			\item  hailstone\_seq( ) $\rightarrow$ 40, 20, 10, 5, 16, 8, 4, 2, 1
		\end{itemize}


\end{enumerate}
\pagebreak
\end{document}