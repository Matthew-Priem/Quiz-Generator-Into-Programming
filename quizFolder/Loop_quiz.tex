\documentclass{article}

\usepackage{amsmath}
\usepackage{amsfonts} % For math fonts.
\usepackage{amssymb} % For other math symbols not covered by amsmath.
\usepackage[pdftex]{graphicx} % For pictures, use \includegraphics[scale=decimal]{pic.png}; must be a .png file type.
\usepackage{multicol}
\usepackage{textcomp}
\usepackage[colorlinks = true, urlcolor = blue]{hyperref}
\usepackage{enumitem}
\usepackage{graphbox} 
\usepackage{subfig}
\usepackage{multicol}
\usepackage{nopageno}


\usepackage{tikz}
\usetikzlibrary{positioning, calc}
\usetikzlibrary{shapes.geometric,angles,quotes}
\usepackage{tikz-3dplot}


%page formatting
\usepackage{fullpage}
\setlength{\parindent}{0pt}


\newcommand{\tab}{\hspace*{0.25in}}
\newcommand{\csq}[1]{\reflectbox{''}#1''}  %This produces CS style quotes.
\newcommand{\csqt}[1]{\text{\reflectbox{''}#1''}}  %This produces CS style quotes as text.


\usepackage{listings}
\lstset
{ %Formatting for code in appendix
    language=Python,
    basicstyle=\footnotesize,
    numbers=left,
    stepnumber=1,
    showstringspaces=false,
    tabsize=2,
    breaklines=true,
    breakatwhitespace=false,
}


\begin{document}



%split_point

%\end{document}
Matt Priem \hfill Loop quiz\\
section 1\\
\begin{enumerate}
	\item 
		Using a loop, write a program that prints every even number 
		between 37 and 1050 (inclusively).


	\item 
		%https://edabit.com/challenge/aqDGJxTYCx7XWyPKc
		Write a program that asks the user for an integer.  Calculate (and then print) the 
		sum of all square numbers up to and including the user's number.

		For example, 
		\begin{itemize}
			\item if user\_number = 3, the result should be 14 since $1^2 + 2^2 + 3^2 = 14$.
			\item if user\_number = 8, the result should be $1^2+2^2+3^2+4^2+5^2+6^2+7^2+8^2=204$.
		\end{itemize}









	\item 
		%https://edabit.com/challenge/xR248CxGSsSrNK5Za
		You are the newest rug fashion designer on the scene, but you're running out of ideas. 
		Write a program that will help you design rugs.  The program should ask for a width, 
		a length, and pattern, and then create a rug consisting of that pattern and dimensions.

		For example, \\ \ \hfill
		\includegraphics[width = 1.5in]{./imgs/rug1.PNG} \hfill  
		\includegraphics[width = 1.5in]{./imgs/rug2.PNG} \hfill \


\end{enumerate}
\pagebreak
Bart Simpson \hfill Loop quiz\\
section 1\\
\begin{enumerate}
	\item 
		Write a program that repeatedly asks the user for integers until a negative integer is 
		given. \\ The program should keep track of the sum of the numbers and print the sum at the 
		end \\(not including the negative number).

		For example, \\ \ \hfill
		\includegraphics[width = 2.in]{./imgs/AddCalc2.PNG} \hfill  
		\includegraphics[width = 2.in]{./imgs/AddCalc1.PNG} \hfill \


	\item 
		%https://edabit.com/challenge/xR248CxGSsSrNK5Za
		You are the newest rug fashion designer on the scene, but you're running out of ideas. 
		Write a program that will help you design rugs.  The program should ask for a width, 
		a length, and pattern, and then create a rug consisting of that pattern and dimensions.

		For example, \\ \ \hfill
		\includegraphics[width = 1.5in]{./imgs/rug1.PNG} \hfill  
		\includegraphics[width = 1.5in]{./imgs/rug2.PNG} \hfill \


	\item 
		Using a loop, write code to calculate the sum of all odd numbers between 50 and 517. 
		Print the result.


\end{enumerate}
\pagebreak
Lone Star \hfill Loop quiz\\
section 2\\
\begin{enumerate}
	\item 
		%https://edabit.com/challenge/aqDGJxTYCx7XWyPKc
		Write a program that asks the user for an integer.  Calculate (and then print) the 
		sum of all square numbers up to and including the user's number.

		For example, 
		\begin{itemize}
			\item if user\_number = 3, the result should be 14 since $1^2 + 2^2 + 3^2 = 14$.
			\item if user\_number = 8, the result should be $1^2+2^2+3^2+4^2+5^2+6^2+7^2+8^2=204$.
		\end{itemize}









	\item 
		Write a program to create a word one letter at a time.  You should prompt the user to enter 
		a single letter one at a time until they type \textit{done}.  Once they type done, output 
		their newly created word.
		
		For example, \\ \ \hfill
		\includegraphics[width = 2.5in]{./imgs/lettersAbcde.PNG} \hfill  
		\includegraphics[width = 2.5in]{./imgs/lettersDexter.PNG} \hfill \


	\item 
		%https://edabit.com/challenge/ksZrMdraPqHjvbaE6
		Write a program that repeatedly asks the user for integers until a negative integer is 
		given.\\  Report back the largest \textbf{even} number the user entered 
		(not including the negative number).  \\
		If the user didn't enter any even numbers report back $-1$.

		For example, \\ \ \hfill
		\includegraphics[height = 1.2in]{./imgs/largestEven1.PNG} \hfill  
		\includegraphics[height = 1.5in]{./imgs/largestEven2.PNG} \hfill  
		\includegraphics[height = 1.2in]{./imgs/largestEven3.PNG} \hfill \


\end{enumerate}
\pagebreak
Dot Matrix \hfill Loop quiz\\
section 3\\
\begin{enumerate}
	\item 
		%https://edabit.com/challenge/6Pf5GGG6HnzbB95gf
		Write code that asks the user for an integer and then prints each number that is a 
		factor of the input.
	
		For example, \\ \ \hfill
		\includegraphics[height = .35in]{./imgs/factors1.PNG} \hfill  
		\includegraphics[height = .35in]{./imgs/factors2.PNG} \hfill  
		\includegraphics[height = .35in]{./imgs/factors3.PNG} \hfill \


	\item 
		%https://edabit.com/challenge/aqDGJxTYCx7XWyPKc
		Write a program that asks the user for an integer.  Calculate (and then print) the 
		sum of all square numbers up to and including the user's number.

		For example, 
		\begin{itemize}
			\item if user\_number = 3, the result should be 14 since $1^2 + 2^2 + 3^2 = 14$.
			\item if user\_number = 8, the result should be $1^2+2^2+3^2+4^2+5^2+6^2+7^2+8^2=204$.
		\end{itemize}









	\item 
		Using a loop, write code to calculate the sum of all odd numbers between 50 and 517. 
		Print the result.


\end{enumerate}
\pagebreak
Alfred Yankovic \hfill Loop quiz\\
section 2\\
\begin{enumerate}
	\item 
		Write a program that repeatedly asks the user for integers until a negative integer is 
		given. \\ The program should keep track of the sum of the numbers and print the sum at the 
		end \\(not including the negative number).

		For example, \\ \ \hfill
		\includegraphics[width = 2.in]{./imgs/AddCalc2.PNG} \hfill  
		\includegraphics[width = 2.in]{./imgs/AddCalc1.PNG} \hfill \


	\item 
		Write a program that asks the user for a word and then, using a loop, prints every other 
		letter of the word starting with the second letter.

		Examples:
		\begin{itemize}
			\item if user\_word = \csq{counterattack}, the result should be \csq{oneatc}
			\item if user\_word = \csq{banana sunday}, the result should be \csq{aaasna}
		\end{itemize}


	\item 
		Write a program to create a word one letter at a time.  You should prompt the user to enter 
		a single letter one at a time until they type \textit{done}.  Once they type done, output 
		their newly created word.
		
		For example, \\ \ \hfill
		\includegraphics[width = 2.5in]{./imgs/lettersAbcde.PNG} \hfill  
		\includegraphics[width = 2.5in]{./imgs/lettersDexter.PNG} \hfill \


\end{enumerate}
\pagebreak
\end{document}