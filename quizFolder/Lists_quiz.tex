\documentclass{article}

\usepackage{amsmath}
\usepackage{amsfonts} % For math fonts.
\usepackage{amssymb} % For other math symbols not covered by amsmath.
\usepackage[pdftex]{graphicx} % For pictures, use \includegraphics[scale=decimal]{pic.png}; must be a .png file type.
\usepackage{multicol}
\usepackage{textcomp}
\usepackage[colorlinks = true, urlcolor = blue]{hyperref}
\usepackage{enumitem}
\usepackage{graphbox} 
\usepackage{subfig}
\usepackage{multicol}
\usepackage{nopageno}


\usepackage{tikz}
\usetikzlibrary{positioning, calc}
\usetikzlibrary{shapes.geometric,angles,quotes}
\usepackage{tikz-3dplot}


%page formatting
\usepackage{fullpage}
\setlength{\parindent}{0pt}


\newcommand{\tab}{\hspace*{0.25in}}
\newcommand{\csq}[1]{\reflectbox{''}#1''}  %This produces CS style quotes.
\newcommand{\csqt}[1]{\text{\reflectbox{''}#1''}}  %This produces CS style quotes as text.


\usepackage{listings}
\lstset
{ %Formatting for code in appendix
    language=Python,
    basicstyle=\footnotesize,
    numbers=left,
    stepnumber=1,
    showstringspaces=false,
    tabsize=2,
    breaklines=true,
    breakatwhitespace=false,
}


\begin{document}



%split_point

%\end{document}
Matt Priem \hfill Lists quiz\\
section 1\\
\begin{enumerate}
	%new
	\item 
		Given a positive integer $n$, the following rules will always create a sequence that 
		ends with 1, called Hailstone Sequence:
		\begin{enumerate}
			\item If $n$ is even, divide by 2
			\item If $n$ is odd, multiply by 3 and add 1 (i.e. $3n+1$)
			\item Continue until $n$ is 1
		\end{enumerate}
		Write a \textbf{function} that returns a list with the hailstone sequence starting at $n$. 
		The argument to the function will be $n$ (the integer to start the sequence from).
		\textbf{Examples:}		
		\begin{itemize}
			\item  hailstone\_seq(25) $\rightarrow$ [25, 76, 38, 19, 58 ... 8, 4, 2, 1], 
			\item  hailstone\_seq(40) $\rightarrow$ [40, 20, 10, 5, 16, 8, 4, 2, 1]
		\end{itemize}



%https://edabit.com/challenge/jwzgYjymYK7Gmro93
	%new
	\item
		Write a \textbf{function} that takes two arguments, a list and a item.  The function should return the indices of all occurrences of the $item$ in the list.

		\textbf{Examples:}		
		\begin{itemize}
			\item  get\_indices( [1, 5, 5, 2, 7], 7) $\rightarrow$ [4]
			\item  get\_indices( [1, 5, 5, 2, 7], 5) $\rightarrow$ [1, 2]
			\item  get\_indices( [1, 5, 5, 2, 7], 8) $\rightarrow$ []
			\item  get\_indices( [\csq{a}, \csq{a}, \csq{b}, \csq{a}, \csq{b}, \csq{a}], \csq{a}) 
				$\rightarrow$ [0, 1, 3, 5]
		\end{itemize}


	%new
	\item 
		Write a \textbf{function} that loops through a word and returns a list with every 
		other letter of the word starting with the \textbf{second} letter.
		The function will take a single argument $word$ (a string representing the word to process).

		\textbf{Examples:}		
		\begin{itemize}
			\item  skip\_letter(\csq{counterattack}) $\rightarrow$ 
				[\csq{o},\csq{n},\csq{e},\csq{a},\csq{t},\csq{c}]
			\item  skip\_letter(\csq{banana sunday}) $\rightarrow$
				[\csq{a},\csq{a},\csq{a},\csq{s},\csq{n},\csq{a}]
		\end{itemize}


\end{enumerate}
\pagebreak
Bart Simpson \hfill Lists quiz\\
section 1\\
\begin{enumerate}
%https://edabit.com/challenge/hYiCzkLBBQSeF8fKF
	%new
	\item 
		In BlackJack, cards are counted with -1, 0, 1 values:
		\begin{itemize}
			\item 2, 3, 4, 5, 6 are counted as +1
			\item 7, 8, 9 are counted as 0
			\item 10, j, q, k, a are counted as -1
		\end{itemize}
		Write a \textbf{function} that takes a list called $cards$, counts the number, 
		and returns it from the list of cards provided.

		\textbf{Examples:}		
		\begin{itemize}
			\item  count([5, 9, 10, 3, \csq{j}, \csq{a}, 4, 8, 5]) $\rightarrow$ 1, 
			\item  count([\csq{a}, \csq{a}, \csq{k}, \csq{q}, \csq{q}, \csq{j}]) $\rightarrow$ -6, 
			\item  count([\csq{a}, 5, 5, 2, 6, 2, 3, 8, 9, 7]) $\rightarrow$ 5
		\end{itemize}



	%new
	\item
		Write a \textbf{function} that takes 3 numbers as arguments, $num\_1$ (first number), 
		$num\_2$ (second number), and $num\_3$ (third number). 
		Return a list of the integers in descending order. 
		You may \textbf{not} use the built-in functions \textit{max}(), \textit{min}(), 
		\textit{sort}(), or \textit{sorted}().
		
	\textbf{Examples:}
	\begin{itemize}
		\item  descending\_order(2, 3, 1) $\rightarrow$ [3, 2, 1], 
		\item  descending\_order(10, 1, 25) $\rightarrow$ [25, 10, 1], 
		\item  descending\_order(2, 45, 4) $\rightarrow$ [45, 4, 2] 
	\end{itemize}



	%new
	\item 
		Write a \textbf{function} that loops through a word and returns a list with every 
		other letter of the word starting with the \textbf{first} letter.
		The function will take a single argument $word$ (a string representing the word to process).

		\textbf{Examples:}		
		\begin{itemize}
			\item  skip\_letter(\csq{counterattack}) $\rightarrow$ 
				[\csq{c},\csq{u},\csq{t},\csq{r},\csq{t},\csq{a},\csq{c}]
			\item  skip\_letter(\csq{banana sunday}) $\rightarrow$
				[\csq{b},\csq{n},\csq{n},\csq{s},\csq{n},\csq{a}]
		\end{itemize}

\end{enumerate}
\pagebreak
Lone Star \hfill Lists quiz\\
section 2\\
\begin{enumerate}
	%new
	\item 
		Write a \textbf{function} that finds the largest odd number in a list $numbers$. Return -1 if not found. 
		You may \textbf{not} use the built-in functions \textit{max}(), \textit{min}(), \textit{sort}(), or \textit{sorted}().

		\textbf{Examples:}		
		\begin{itemize}
			\item  largest\_odd([3, 7, 2, 1, 7, 9, 10, 13]) $\rightarrow$ 13,
			\item  largest\_odd([2, 4, 6, 8]) $\rightarrow$ -1,
			\item  largest\_odd([0, 19, 18973623]) $\rightarrow$ 18973623
		\end{itemize}

	%new
	\item
		Write a \textbf{function} that takes 3 numbers as arguments, $num\_1$ (first number), 
		$num\_2$ (second number), and $num\_3$ (third number). 
		Return a list of the integers in descending order. 
		You may \textbf{not} use the built-in functions \textit{max}(), \textit{min}(), 
		\textit{sort}(), or \textit{sorted}().
		
	\textbf{Examples:}
	\begin{itemize}
		\item  descending\_order(2, 3, 1) $\rightarrow$ [3, 2, 1], 
		\item  descending\_order(10, 1, 25) $\rightarrow$ [25, 10, 1], 
		\item  descending\_order(2, 45, 4) $\rightarrow$ [45, 4, 2] 
	\end{itemize}



	%new
	\item 
		Write a \textbf{function} that loops through and returns a list with every odd number between two
		integers (inclusive). The arguments to the function will be $smaller\_num$ and 
		$larger\_num$.

		\textbf{Examples:}		
		\begin{itemize}
			\item  output\_odd(37, 1050) $\rightarrow$ [37, 39, 41, \dots, 1049], 
			\item  output\_odd(1, 2000) $\rightarrow$ [1, 3, 5, \dots, 1999], 
			\item  output\_odd(50, 199) $\rightarrow$ [51, 53, 55, \dots, 197, 199]
		\end{itemize}


\end{enumerate}
\pagebreak
Dot Matrix \hfill Lists quiz\\
section 3\\
\begin{enumerate}
	%new
	\item
		Write a \textbf{function} that takes 3 numbers as arguments, $num\_1$ (first number), 
		$num\_2$ (second number), and $num\_3$ (third number). 
		Return a list of the integers in ascending order. 
		You may \textbf{not} use the built-in functions \textit{max}(), \textit{min}(), 
		\textit{sort}(), or \textit{sorted}().
		
	\textbf{Examples:}
	\begin{itemize}
		\item  ascending\_order(2, 3, 1) $\rightarrow$ [1, 2, 3], 
		\item  ascending\_order(10, 1, 25) $\rightarrow$ [1, 10, 25], 
		\item  ascending\_order(2, 45, 4) $\rightarrow$ [2, 4, 45] 
	\end{itemize}


	%new
	\item
		Write a \textbf{function} that takes 3 numbers as arguments, $num\_1$ (first number), 
		$num\_2$ (second number), and $num\_3$ (third number). 
		Return a list of the integers in descending order. 
		You may \textbf{not} use the built-in functions \textit{max}(), \textit{min}(), 
		\textit{sort}(), or \textit{sorted}().
		
	\textbf{Examples:}
	\begin{itemize}
		\item  descending\_order(2, 3, 1) $\rightarrow$ [3, 2, 1], 
		\item  descending\_order(10, 1, 25) $\rightarrow$ [25, 10, 1], 
		\item  descending\_order(2, 45, 4) $\rightarrow$ [45, 4, 2] 
	\end{itemize}



	%new
	\item
		To train for an upcoming marathon, Samuel goes on one long-distance run each Saturday. 
		He wants to track how often the number of miles he runs fall short of the previous Saturday. 
		This is called a lag day. Write a \textbf{function} that takes in a list of miles 
		run every Saturday and returns Samuel's total number of lag days.

		\textbf{Examples:}		
		\begin{itemize}
			\item  lag\_days([5, 3, 2, 1]) $\rightarrow$ 3, 
				(3 lag days, day2 since (3$<$5), day3 since (2$<$3), and day4 since (1$<$2))
			\item  lag\_days([10, 11, 12, 9, 10]) $\rightarrow$ 1, 
			\item  lag\_days([6, 5, 4, 3, 2, 9]) $\rightarrow$ 4, 
			\item  lag\_days([9, 9]) $\rightarrow$ 0
		\end{itemize}

\end{enumerate}
\pagebreak
Alfred Yankovic \hfill Lists quiz\\
section 2\\
\begin{enumerate}
%https://edabit.com/challenge/egMp3GWyN8TPptbZA
	%new
	\item
		YouTube currently displays a like and a dislike button, allowing you to express your opinions about particular content. 
		It's set up in such a way that you cannot like and dislike a video at the same time.
		There are two other interesting rules to be noted about the interface:
		\begin{enumerate}
			\item Pressing a button, which is already active, will undo your press.
			\item If you press the like button after pressing the dislike button, the like button overwrites the previous \csq{dislike} state. The same is true for the other way round.
		\end{enumerate}
		Write a \textbf{function} that takes in a list of button inputs $events$ and returns the final state.

		\textbf{Examples:}		
		\begin{itemize}
			\item  like\_or\_dislike([\csq{dislike}]) $\rightarrow$ \csq{dislike},
			\item  like\_or\_dislike([\csq{like}, \csq{like}]) $\rightarrow$ \csq{nothing},
			\item  like\_or\_dislike([\csq{dislike}, \csq{like}]) $\rightarrow$ \csq{like},
			\item  like\_or\_dislike([\csq{like}, \csq{dislike}, \csq{dislike}]) $\rightarrow$ \csq{nothing},
		\end{itemize}




%https://edabit.com/challenge/2yHQwkecEHZBfHcmN
	%new
	\item
		To train for an upcoming marathon, Johnny goes on one long-distance run each Saturday. 
		He wants to track how often the number of miles he runs exceeds the previous Saturday. 
		This is called a progress day. Write a \textbf{function} that takes in a list of miles 
		run every Saturday and returns Johnny's total number of progress days.

		\textbf{Examples:}		
		\begin{itemize}
			\item  progress\_days([3, 4, 1, 2]) $\rightarrow$ 2, 
				(Two progress days, day 2 since $(4>3)$ and day 4 since $(2>1)$)
			\item  progress\_days([10, 11, 12, 9, 10]) $\rightarrow$ 3, 
			\item  progress\_days([6, 5, 4, 3, 2, 9]) $\rightarrow$ 1, 
			\item  progress\_days([9, 9]) $\rightarrow$ 0
		\end{itemize}

	%new
	\item 
		Write a \textbf{function} that loops through and returns a list with every odd number between two
		integers (inclusive). The arguments to the function will be $smaller\_num$ and 
		$larger\_num$.

		\textbf{Examples:}		
		\begin{itemize}
			\item  output\_odd(37, 1050) $\rightarrow$ [37, 39, 41, \dots, 1049], 
			\item  output\_odd(1, 2000) $\rightarrow$ [1, 3, 5, \dots, 1999], 
			\item  output\_odd(50, 199) $\rightarrow$ [51, 53, 55, \dots, 197, 199]
		\end{itemize}


\end{enumerate}
\pagebreak
\end{document}